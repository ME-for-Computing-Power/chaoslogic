\documentclass[UTF8]{ctexart}

\usepackage{geometry}
%\geometry{a4paper, left=2.5cm, right=2.5cm, top=3cm, bottom=3cm}

%\usepackage{fontspec}

\usepackage{graphicx}
\usepackage{svg}
\usepackage{hyperref}
\hypersetup{
    colorlinks=true,
    linkcolor=blue,
    urlcolor=blue,
    citecolor=blue
}

\title{验证自动化流程设计说明}
\author{孙林涵,诸人豪,张煜}
\date{\today}

\begin{document}

\maketitle

\begin{abstract}
本文档主要介绍本软件的设计思路、实现方法及相关注意事项。
\end{abstract}

\section{介绍}
%简要介绍文档背景和目的。
本项目设计了一套由大语言模型(Large Language Model, LLM)驱动的RTL代码的生成与验证自动化软件,旨在提高IC设计的效率和准确性。

\section{Agent设计}
软件通过设置项目管理、设计工程师和验证工程师三个Agent,模仿一般的IC设计流程,使通用LLM
生成可靠的RTL代码。下面,介绍各个Agent的设计。

\subsection{项目工程师Agent}
项目工程师解读用户的需求,生成项目的技术规范(下文简称Spec),并将其存储在文件中。

当设计完成时,项目工程师还会根据验证工程师的验证报告,检查设计是否满足规范要求。


\subsection{设计工程师Agent}
设计工程师根据项目工程师提供的Spec,生成RTL代码。设计工程师会将生成的代码存储在文件中,并在必要时进行修改。为避免LLM生成的代码存在语法错误,每次设计工程师提交代码时,都会使用VCS的vlogan工具进行语法检查。若检查出语法错误,设计工程师会根据错误信息进行修改,直到消除所有报错为止。

另外,设计工程师还会根据验证工程师的反馈,修改RTL代码以满足验证需求。

\subsection{验证工程师Agent}
验证工程师根据项目工程师提供的Spec与用户提供的验证方案,生成验证计划,并编写测试用例。验证工程师会使用VCS编译仿真程序,并运行测试用例,生成验证报告。同样,当编译仿真程序时,若出现编译错误,验证工程师也会根据错误信息进行修改,直至成功编译得到simv仿真程序为止。
接下来,验证工程师会检查simv仿真程序的输出,判断RTL代码是否存在问题。当RTL代码存在问题时,验证工程师会将问题反馈给设计工程师,并要求其修改RTL代码。当验证工程师认为RTL代码满足验证需求时,会将验证报告提交给项目工程师。


\section{运行流程}
接下来,介绍本软件是如何调用各个Agent以完成RTL代码的生成与验证的。

%流程图
\begin{figure}[htbp]
    \centering
    \includesvg[width=0.8\textwidth]{flow.drawio.svg}
    \caption{程序流程图}
    \label{fig:flowchart}
\end{figure}

\section{注意事项}
列举在设计和实现过程中需要注意的问题。

\section{结论}
总结全文,提出后续改进方向。

\end{document}