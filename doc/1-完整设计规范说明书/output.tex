\documentclass[12pt,]{article}
\usepackage{geometry} 		% 設定邊界
\geometry{
  top=1in,
  inner=1in,
  outer=1in,
  bottom=1in,
  headheight=3ex,
  headsep=2ex
}
\usepackage{booktabs}
\usepackage[T1]{fontenc}
\usepackage{lmodern}
\usepackage{amssymb,amsmath}
\usepackage{ifxetex,ifluatex}
\usepackage{fixltx2e} % provides \textsubscript
\providecommand{\tightlist}{%
  \setlength{\itemsep}{0pt}\setlength{\parskip}{0pt}}
% use upquote if available, for straight quotes in verbatim environments
\IfFileExists{upquote.sty}{\usepackage{upquote}}{}
\ifnum 0\ifxetex 1\fi\ifluatex 1\fi=0 % if pdftex
  \usepackage[utf8]{inputenc}
\else % if luatex or xelatex
  \usepackage{fontspec} 	% 允許設定字體
  \usepackage{xeCJK} 		% 分開設置中英文字型
  \setCJKmainfont{Noto Serif CJK SC} 	% 設定中文字型
  \setmainfont{Georgia} 	% 設定英文字型
  \setromanfont{Georgia} 	% 字型
  \setmonofont{Courier New}
  \linespread{1.2}\selectfont 	% 行距
  \XeTeXlinebreaklocale "zh" 	% 針對中文自動換行
  \XeTeXlinebreakskip = 0pt plus 1pt % 字與字之間加入0pt至1pt的間距,確保左右對整齊
  \parindent 0em 		% 段落縮進
  \setlength{\parskip}{20pt} 	% 段落之間的距離
  \ifxetex
    \usepackage{xltxtra,xunicode}
  \fi
  \defaultfontfeatures{Mapping=tex-text,Scale=MatchLowercase}
  \newcommand{\euro}{€}
    \setmainfont{Noto Serif CJK SC}
    \setmonofont{Noto Sans Mono CJK SC}
\fi
% use microtype if available
\IfFileExists{microtype.sty}{\usepackage{microtype}}{}
\usepackage{longtable}
\ifxetex
  \usepackage[setpagesize=false, % page size defined by xetex
              unicode=false, % unicode breaks when used with xetex
              xetex]{hyperref}
\else
  \usepackage[unicode=true]{hyperref}
\fi
\hypersetup{breaklinks=true,
            bookmarks=true,
            pdfauthor={},
            pdftitle={},
            colorlinks=true,
            urlcolor=blue,
            linkcolor=magenta,
            pdfborder={0 0 0}}
\urlstyle{same}  % don't use monospace font for urls
\setlength{\parindent}{0pt}
%\setlength{\parskip}{6pt plus 2pt minus 1pt}
\setlength{\emergencystretch}{3em}  % prevent overfull lines

\title{\huge 在OSX平台上的XeLaTeX中文測試} % 設置標題,使用巨大字體
\author{FoolEgg.com} 		% 設置作者
\date{February 2013} 		% 設置日期
\usepackage{titling}
\setlength{\droptitle}{-8em} 	% 將標題移動至頁面的上面

\usepackage{fancyhdr}
\usepackage{lastpage}
\usepackage{svg}
\pagestyle{fancyplain}

\setcounter{secnumdepth}{0}

\author{}
\date{}

\begin{document}
\hypertarget{ux5e27ux683cux5f0fux5e8fux5217ux68c0ux6d4bux751fux6210ux6a21ux5757ux8bbeux8ba1ux89c4ux8303}{%
\section{帧格式序列检测生成模块设计规范}\label{ux5e27ux683cux5f0fux5e8fux5217ux68c0ux6d4bux751fux6210ux6a21ux5757ux8bbeux8ba1ux89c4ux8303}}

\tableofcontents
\newpage

\hypertarget{ux8bbeux8ba1ux6982ux8ff0}{%
\subsection{设计概述}\label{ux8bbeux8ba1ux6982ux8ff0}}

本规范文件描述了帧格式序列检测与生成模块(Frame Format Sequence
Generator)的设计规范和特性要求,该模块旨在检测输入数据流中的特定帧格式,并在检测到目标帧时进行串行输出。

\begin{figure}[h!]
  \includesvg[width = 0.99\textwidth]{schematic.svg}
\end{figure}

如上框图所示,模块接收到数据输入 \texttt{data\_in}
后,在模块进行以下处理过程:

\begin{enumerate}
\def\labelenumi{\arabic{enumi}.}
\item
  输入数据检测: 模块接收并采样输入数据。
\item
  帧格式识别:
  检测并识别输入数据中的特定帧格式,包括帧头、通道选择字段、数据、CRC
  校验字段和帧尾 5 个部分。
\item
  解帧与数据提取: 识别帧头、通道选择和帧尾,提取数据部分。
\item
  CRC 校验: 对提取的数据进行 CRC 校验,确保数据完整性。

  \begin{itemize}
  \item
    如果 CRC 校验成功,数据将被写入异步 FIFO 进行缓存。
  \item
    如果 CRC 校验失败,模块报告错误并丢弃错误数据。
  \end{itemize}
\item
  异步 FIFO 缓存: 提取的数据通过异步 FIFO 进行缓存,提供 FIFO 空、FIFO
  满信号状态指示。
\item
  数据编码: 从 FIFO 中读取数据,按照格雷码进行编码。
\item
  根据通道选择进行串行输出:
  编码后的数据根据通道选择的数值,决定在哪个通道进行串行输出
\end{enumerate}

\hypertarget{ux5b50ux6a21ux5757ux6982ux8ff0}{%
\subsection{子模块概述}\label{ux5b50ux6a21ux5757ux6982ux8ff0}}

\hypertarget{input_stage}{%
\subsubsection{input\_stage}\label{input_stage}}

整体说明:该模块旨在检测输入数据流中的特定帧格式,并提取有效数据。每帧输入16位数据,输出5种状态的指示信号。具体帧格式如下:

\begin{itemize}
\item
  帧头: \texttt{32} 位,取值为 \texttt{E0E0E0E0}
\item
  通道选择: \texttt{8} 位, 独热码,高位数据丢弃
\item
  数据: \texttt{N} 位(可变长度,限制在\texttt{16} 位到\texttt{128}
  位之间,是\texttt{16}的整数倍,按照 Big-Endian 方式输入)
\item
  CRC 校验字段: \texttt{16} 位
\item
  帧尾: \texttt{32} 位,取值为 \texttt{0E0E0E0E}
\end{itemize}

\textbf{状态机状态定义}:

\texttt{1.\hspace{0pt}IDLE\hspace{0pt}(空闲状态)}

\begin{itemize}
\item
  初始状态,等待帧头起始
\item
  行为:持续监测输入数据是否为帧头的第一部分 E0E0
\item
  转换条件:当 data\_in == 16'hE0E0 时进入 HEAD\_CHECK
\end{itemize}

\texttt{2.HEAD\_CHECK\hspace{0pt}(帧头校验状态)}

\begin{itemize}
\item
  验证完整的32位帧头
\item
  行为:检查第二个16位数据是否为 E0E0
\item
  转换条件: 若 data\_in == 16'hE0E0 进入 CHANNEL 否则返回 IDLE
\end{itemize}

\texttt{3.\hspace{0pt}CHANNEL\hspace{0pt}(通道选择状态)}

\begin{itemize}
\item
  读取通道选择字段
\item
  行为:存储输入数据的低8位(高8位丢弃)到 data\_ch 寄存器
\item
  转换条件:无条件进入 DATA 状态(仅需1周期)
\end{itemize}

\texttt{4.\hspace{0pt}DATA\hspace{0pt}(数据接收状态)}

\begin{itemize}
\item
  接收数据字段和后续字段
\item
  行为:

  启动4位计数器(\texttt{data\_counte}r)从0开始计数

  每个周期将32位寄存器
  \texttt{tail\_detec\_reg}(用于帧尾检测,初始化全0)左移16位,并存入输入数据\texttt{data\_in}

  每个周期将160位寄存器
  \texttt{full\_data\_reg}(用于存储数据字段,初始化全0)左移16位,并存入输入数据\texttt{data\_in}

  计数器递增,溢出时丢弃数据
\item
  转换条件: 当检测到\texttt{data\_in}为 \texttt{0E0E0E0E} 时进入
  \texttt{CRC\_OUTPUT},停止寄存器\texttt{tail\_detec\_reg}和\texttt{full\_data\_reg}的存储
  若计数器溢出(≥15)返回 \texttt{IDLE}
\end{itemize}

\texttt{5.\hspace{0pt}CRC\_OUTPUT\hspace{0pt}(CRC提取状态)}

\begin{itemize}
\item
  提取CRC字段并计算数据长度
\item
  行为:

  从32位寄存器\texttt{tail\_detec\_reg}中获取CRC字段(帧尾前1周期的数据)

  计算数据长度:data\_count = data\_counter - 2

  截取 full\_data\_reg 的高128位作为 data\_128
\item
  转换条件:

  当检测到data\_in为 0E0E0 时进入
  ENABLE\_CRC(确保帧尾正确)否则数据错误,返回 IDLE
\end{itemize}

\texttt{6.\hspace{0pt}ENABLE\_CRC\hspace{0pt}(CRC校验使能状态)}

\begin{itemize}
\item
  使能CRC校验模块
\item
  行为: 拉高 start\_crc信号,启动CRC校验模块,结果由CRC校验模块返回
\item
  转换条件: \textbf{无条件}返回 IDLE
\end{itemize}

单独为与CRC交互编写一个always块,确保后续数据的输入在CRC校验期间不会被丢弃:

\begin{itemize}
\item
  行为:

  接收到start\_crc信号后,拉低start\_crc信号,crc\_cnt开始计数。

  当crc\_cnt小于data\_count的值的时候,维持 crc16\_valid
  高电平,同时每个周期从data\_128低位开始每16位作为data\_to\_crc发送给CRC校验模块。所有数据已发送后清零crc16\_valid和crc\_cnt。

  接收到crc16\_done信号后,检查CRC结果data\_from\_crc和crc\_field\_reg是否相等。相等时候拉高fifo\_w\_enable,拉低crc\_err,data\_to\_fifo
  赋值为\{data\_128, data\_ch,
  data\_count\}写入fifo。不相等时拉低fifo\_w\_enable,拉高crc\_err。
\end{itemize}

\textbf{顶层IO}

\begin{longtable}[]{@{}lll@{}}
\toprule\noalign{}
信号 & 位宽 & I/O \\
\midrule\noalign{}
\endhead
\bottomrule\noalign{}
\endlastfoot
clk\_in & 1 & I \\
rst\_n & 1 & I \\
data\_in & 16 & I \\
crc\_err & 1 & O \\
data\_to\_fifo & 140 & O \\
fifo\_w\_enable & 1 & O \\
data\_to\_crc & 16 & O \\
data\_from\_crc & 16 & I \\
crc16\_done & 1 & I \\
crc16\_valid & 1 & O \\
\end{longtable}

\textbf{信号说明}

\begin{itemize}
\tightlist
\item
  \texttt{clk\_in}: 输入时钟
\item
  \texttt{rst\_n}:异步复位信号,低电平有效
\item
  \texttt{data\_in}: 16位并行输入数据
\item
  \texttt{crc\_err}: CRC校验结果,错误时为真
\item
  \texttt{data\_to\_fifo}: 输出到fifo的数据
\item
  \texttt{fifo\_w\_enable}: fifo模块的写使能信号
\item
  \texttt{data\_to\_crc}:输入到CRC校验模块的待校验数据
\item
  \texttt{data\_from\_crc}:从CRC校验模块返回的数据校验值
\item
  \texttt{crc16\_done}:从CRC校验模块返回的数据校验值有效信号,高电平有效
\item
  \texttt{crc16\_valid}:发送给CRC校验模块的使能信号,高电平有效
\end{itemize}

\hypertarget{async_fifo}{%
\subsubsection{async\_fifo}\label{async_fifo}}

整体说明: 传统的异步fifo,宽度为140位,深度为8,提供 fifo 空、fifo
满信号状态指示

\textbf{顶层IO}

\begin{longtable}[]{@{}lll@{}}
\toprule\noalign{}
信号 & 位宽 & I/O \\
\midrule\noalign{}
\endhead
\bottomrule\noalign{}
\endlastfoot
clk\_in & 1 & I \\
clk\_out & 1 & I \\
rst\_n & 1 & I \\
fifo\_w\_enable & 1 & I \\
fifo\_r\_enable & 1 & I \\
data\_to\_fifo & 140 & I \\
data\_from\_fifo & 140 & O \\
fifo\_empty & 1 & O \\
fifo\_full & 1 & O \\
\end{longtable}

\textbf{信号说明}

\texttt{clk\_in}:fifo输入时钟域的时钟

\texttt{clk\_out}: fifo输出时钟域的时钟,与 \texttt{clk\_in}
为同频异步。

\texttt{rst\_n}:异步复位信号,低电平有效

\texttt{fifo\_w\_enable}:fifo写使能信号

\texttt{fifo\_r\_enable}:fifo读使能信号

\texttt{data\_to\_fifo}:fifo输入数据

\texttt{data\_from\_fifo}:fifo输出数据,当\texttt{fifo\_empty}为1时应读出140`b0

\texttt{fifo\_empty}:fifo空信号

\texttt{fifo\_full}:fifo满信号

\textbf{时序说明}

数据将在\texttt{fifo\_w\_enable}被拉高的时钟周期内被写入

数据将在\texttt{fifo\_r\_enable}被拉高的时钟周期内被读出

\hypertarget{fifo_data_resolu}{%
\subsubsection{fifo\_data\_resolu}\label{fifo_data_resolu}}

整体说明:纯组合逻辑模块,将从fifo读到的140位数据分离为128位数据位+8位通道选择位+4位数据长度表示位。
其中高128位位数据位,中间8位位通道选择位,低4位为数据长度表示位。

4位数据长度表示位生成16位具体数据长度位\texttt{data\_count},例如0-8分别输出0
16 32 48 64 80 96 112 128。

数据位高位在前,长度由\texttt{data\_count}给出,不足128位的数据将在低位补0。
数据位转为格雷码后,在低位补0补齐128位,输出信号\texttt{data\_gray}也是高位在前。

8位通道选择位不做处理,直接输出8位信号\texttt{vld\_ch}。

\textbf{顶层IO}

\begin{longtable}[]{@{}lll@{}}
\toprule\noalign{}
信号 & 位宽 & I/O \\
\midrule\noalign{}
\endhead
\bottomrule\noalign{}
\endlastfoot
data\_from\_fifo & 140 & I \\
data\_gray & 128 & O \\
vld\_ch & 8 & O \\
data\_count & 16 & O \\
\end{longtable}

\textbf{信号说明}

\texttt{data\_from\_fifo}:fifo输出数据
\texttt{data\_gray}:输出数据的格雷码表示 \texttt{vld\_ch}:通道选择数据
\texttt{data\_count}:具体数据长度位

\hypertarget{output_stage}{%
\subsubsection{output\_stage}\label{output_stage}}

整体说明:本模块根据前面模块提供的信息,将数据发送到对应的通道并转为串行信号输出。根据\texttt{vld\_ch}所输入的独热码,将输入的\texttt{data\_gray}发送到对应的数据串行输出通道
(\texttt{data\_out\_ch1\textasciitilde{}8}),同时拉高对应通道的数据有效信号
(\texttt{data\_vld\_ch1\textasciitilde{}8})。根据\texttt{data\_count},具体决定输出\texttt{data\_vld\_ch}的持续周期数,即\texttt{data\_vld\_ch}仅在输出时拉高。

\textbf{顶层IO}

\begin{longtable}[]{@{}lll@{}}
\toprule\noalign{}
信号 & 位宽 & I/O \\
\midrule\noalign{}
\endhead
\bottomrule\noalign{}
\endlastfoot
rst\_n & 1 & I \\
clk\_out16x & 1 & I \\
data\_gray & 128 & I \\
vld\_ch & 8 & I \\
data\_count & 16 & I \\
crc\_valid & 1 & O \\
data\_out\_ch1 & 1 & O \\
data\_out\_ch2 & 1 & O \\
\ldots{} & 1 & O \\
data\_out\_ch8 & 1 & O \\
data\_vld\_ch1 & 1 & O \\
data\_vld\_ch2 & 1 & O \\
\ldots{} & 1 & O \\
data\_vld\_ch8 & 1 & O \\
\end{longtable}

\textbf{信号说明}

\begin{itemize}
\tightlist
\item
  \texttt{rst\_n}:异步复位信号,低电平有效
\item
  \texttt{clk\_out16x}:输出串行信号速率的时钟
\item
  \texttt{data\_gray}:输出数据的格雷码表示
\item
  \texttt{vld\_ch}:通道选择数据,8位宽独热码
\item
  \texttt{data\_count}:\texttt{data\_gray}的具体数据长度
\item
  \texttt{crc\_valid}:任一\texttt{data\_vld\_ch*}拉高时即拉高
\item
  \texttt{data\_out\_ch1\textasciitilde{}8}:数据串行输出通道,输出时高位在前
\item
  \texttt{data\_vld\_ch1\textasciitilde{}8}:数据有效信号通道,仅在输出时拉高
\end{itemize}

\hypertarget{crc_calcu}{%
\subsubsection{crc\_calcu}\label{crc_calcu}}

设计一个CRC校验模块,采用CRC-16/CCITT算法,多项式:0x1021,初始值:为输入的crc信号,输入不反转,输出不反转,输出异或值:0x0000。

\textbf{顶层IO}

\begin{longtable}[]{@{}lll@{}}
\toprule\noalign{}
信号 & 位宽 & I/O \\
\midrule\noalign{}
\endhead
\bottomrule\noalign{}
\endlastfoot
data\_to\_crc & 16 & I \\
crc & 16 & I \\
data\_from\_crc & 16 & O \\
\end{longtable}

\textbf{信号说明}


\texttt{data\_to\_crc}:从帧格式识别模块输入的待校验数据

\texttt{crc}:从帧格式识别模块输入的crc校验初始值

\texttt{data\_from\_crc}:输出数据的CRC校验值

\end{document}
